\fio\\
\position{} \kafedra

\vspace{2em}

\begin{center}\bf
УЧЕБНАЯ РАБОТА НА \year{} УЧЕБНЫЙ ГОД
\end{center}

\hbox to \textwidth {\hbox{Рассмотрен}\hfil\hbox{УТВЕРЖДАЮ}}
\hbox to \textwidth {\hbox{на заседании кафедры}\hfil\hbox{}}
\hbox to \textwidth {\hbox{\protokol}\hfil\hbox{Зав. кафедрой\und}}

\vspace{2em}

\newcounter{issue_no} % Текущий номер пункта
\newcounter{inline} % Часы в строке

% На семестр
\newcounter{cntlek} % Часы в столбце лекций
\newcounter{cntsem} % Часы в столбце практик
\newcounter{cntlab} % Часы в столбце лабораторных
\newcounter{cntzac} % Часы в столбце зачётов
\newcounter{cntekz} % Часы в столбце экзаменов
\newcounter{cntkon} % Часы в столбце консультаций
\newcounter{cntsam} % Часы в столбце проверки сам. работы
\newcounter{cntind} % Часы в столбце контрольных работ
\newcounter{cntpra} % Часы в столбце практики
\newcounter{cntxxx} % Часы в столбце прочего
\newcounter{cnttotal} % Общее число часов


\def\resetcnt{
\setcounter{issue_no}{0} % Часы в столбце лекций
\setcounter{cntlek}{0} % Часы в столбце лекций
\setcounter{cntsem}{0} % Часы в столбце практик
\setcounter{cntlab}{0} % Часы в столбце лабораторных
\setcounter{cntzac}{0} % Часы в столбце зачётов
\setcounter{cntekz}{0} % Часы в столбце экзаменов
\setcounter{cntkon}{0} % Часы в столбце консультаций
\setcounter{cntsam}{0} % Часы в столбце проверки сам. работы
\setcounter{cntind}{0} % Часы в столбце контрольных работ
\setcounter{cntpra}{0} % Часы в столбце практики
\setcounter{cntxxx}{0} % Часы в столбце прочего
\setcounter{cnttotal}{0} % Часы в столбце итого
\setcounter{mini}{15} % Максимальное число пунктов (остаток выводится пустыми строками)
}

% Для фактических часов, можно несколько раз писать \lekc{3}{\lekc{4}...
% см. ниже
\newcounter{rowlek} % Часы в строке лекций
\newcounter{rowsem} % Часы в строке практик
\newcounter{rowlab} % Часы в строке лабораторных
\newcounter{rowzac} % Часы в строке зачётов
\newcounter{rowekz} % Часы в строке экзаменов
\newcounter{rowkon} % Часы в строке консультаций
\newcounter{rowsam} % Часы в строке проверки сам. работы
\newcounter{rowind} % Часы в строке контрольных работ
\newcounter{rowpra} % Часы в строке практики
\newcounter{rowxxx} % Часы в строке прочего

\newcounter{yearlek} % Часы лекций в год (фактически)
\newcounter{yearsem} % Часы практик в год (фактически)
\newcounter{yearlab} % Часы лабораторных в год (фактически)
\newcounter{yearzac} % Часы зачётов в год (фактически)
\newcounter{yearekz} % Часы экзаменов в год (фактически)
\newcounter{yearkon} % Часы консультаций в год (фактически)
\newcounter{yearsam} % Часы проверки сам. работы в год (фактически)
\newcounter{yearind} % Часы контрольных работ в год (фактически)
\newcounter{yearpra} % Часы практики в год (фактически)
\newcounter{yearxxx} % Часы прочего в год (фактически)
\newcounter{yeartotal} % Часы всего в год (фактически)

% Команды прибавления часов
\def\lekc#1{\setcounter{rowlek}{#1}\addtocounter{cntlek}{#1}}
\def\semprak#1{\setcounter{rowsem}{#1}\addtocounter{cntsem}{#1}}
\def\labor#1{\setcounter{rowlab}{#1}\addtocounter{cntlab}{#1}}
\def\zacjot#1{\setcounter{rowzac}{#1}\addtocounter{cntzac}{#1}}
\def\ekzam#1{\setcounter{rowekz}{#1}\addtocounter{cntekz}{#1}}
\def\konsul#1{\setcounter{rowkon}{#1}\addtocounter{cntkon}{#1}}
\def\samost#1{\setcounter{rowsam}{#1}\addtocounter{cntsam}{#1}}
\def\indiv#1{\setcounter{rowind}{#1}\addtocounter{cntind}{#1}}
\def\praktika#1{\setcounter{rowpra}{#1}\addtocounter{cntpra}{#1}}
\def\xxx#1{\setcounter{rowxxx}{#1}\addtocounter{cntxxx}{#1}}

% Показать, если не ноль
\def\showcnt#1{\ifnum\value{#1}>0{\arabic{#1}}\fi}

% Показать, если не ноль; добавить в общий счётчик строки
\def\collect#1{\addtocounter{inline}{\value{#1}}\showcnt{#1}\setcounter{#1}{0}}

% Переход на след. строку, обнуляем счётчики
\def\total{
\collect{rowlek} &
\collect{rowsem} &
\collect{rowlab} &
\collect{rowzac} &
\collect{rowekz} &
\collect{rowkon} &
\collect{rowind} &
\collect{rowsam} &
\collect{rowpra} &
\collect{rowxxx} &
\addtocounter{cnttotal}{\value{inline}}%
\bf\showcnt{inline}\setcounter{inline}{0} \\
\hline
}
\let\together=\total
\let\itogo=\total

% Начало строки в плане, параметры: название курса, у кого курс
\def\plan#1#2{
\addtocounter{issue_no}{1}
\arabic{issue_no} & \small #1 & #2 &%
}

% Начало строки в фактическом выполнении, параметр: название месяца
\def\fact#1{\multicolumn{3}{|l|}{#1} &}
\let\fakt=\fact


% Эталонная строка для указания ширины столбцов
% первые три числа - лексии, практики, лабораторные - их много =)
\def\phantomrow{%
15 & \small Математический анализ & 23 физики & \bf 468 & \bf 468 &
\bf 168 & \bf 68 & \bf 68 & \bf 68 & \bf 68 & \bf 68 & \bf 68 & \bf 68 & \bf 968%
}

\def\emptyrow{&   &   &   &   &   &   &   &   &   &   &   &   &}

% Параметры: семестр, план, фактически
\def\tables#1#2#3{%
\vbox{
\resetcnt
\begin{longtable}{|c|l|c|c|c|c|c|c|c|c|c|c|c|c|}
\hline
\raisebox{2em}{\textnumero} &
\raisebox{2em}{\centering\twolines{Дисциплина,}{форма занятия}} &
\begin{sideways}\twolines{Факультет,}{курс, группа}\end{sideways} &
\begin{sideways}Лекции \end{sideways} &
\begin{sideways}Сем. и прак. \end{sideways} &
\begin{sideways}Лабор. \end{sideways} &
\begin{sideways}Зачёты \end{sideways} &
\begin{sideways}Экзамены \end{sideways} &
\begin{sideways}Консульт. \end{sideways} &
\begin{sideways}Индив. раб. \end{sideways} &
\begin{sideways}Самост. раб. \end{sideways} &
\begin{sideways}Практика \end{sideways} &
\begin{sideways}Прочее \end{sideways} &
\begin{sideways}\bf Итого \end{sideways} \\
\hline

\tiny1&\tiny\hfil2\hfil&\tiny3&\tiny4&\tiny5&\tiny6&\tiny7&\tiny8&\tiny9&\tiny10&\tiny11&\tiny12&\tiny13&\tiny14\\
\hline
\hline

\phantomrow \kill

#2

\addtocounter{mini}{-\value{issue_no}}
\replicate{\value{mini}}{\emptyrow\\ \hline}

\multicolumn{3}{|l|}{\bf Итого на #1-й семестр:} &
\bf\showcnt{cntlek} &
\bf\showcnt{cntsem} &
\bf\showcnt{cntlab} &
\bf\showcnt{cntzac} &
\bf\showcnt{cntekz} &
\bf\showcnt{cntkon} &
\bf\showcnt{cntind} &
\bf\showcnt{cntsam} &
\bf\showcnt{cntpra} &
\bf\showcnt{cntxxx} &
\showcnt{cnttotal}\\

\hline

\end{longtable}
\addtocounter{cntuchebtotal}{\value{cnttotal}} % глобальный счётчик учебной нагрузки (plan.tex)
}

\vfil

\vbox{
\resetcnt
\begin{longtable}{|c|l|c|c|c|c|c|c|c|c|c|c|c|c|}

\phantomrow \kill

\multicolumn{14}{l}{{\bf Выполнено фактически за #1-й семестр:}}\\

%\multicolumn{3}{|l|}{}&\tiny4&\tiny5&\tiny6&\tiny7&\tiny8&\tiny9&\tiny10&\tiny11&\tiny12&\tiny13&\tiny14\\
\hline

#3

\multicolumn{3}{|l|}{\bf Итого за #1-й семестр:} &
\bf\showcnt{cntlek}\addtocounter{yearlek}{\value{cntlek}}&
\bf\showcnt{cntsem}\addtocounter{yearsem}{\value{cntsem}}&
\bf\showcnt{cntlab}\addtocounter{yearlab}{\value{cntlab}}&
\bf\showcnt{cntzac}\addtocounter{yearzac}{\value{cntzac}}&
\bf\showcnt{cntekz}\addtocounter{yearekz}{\value{cntekz}}&
\bf\showcnt{cntkon}\addtocounter{yearkon}{\value{cntkon}}&
\bf\showcnt{cntind}\addtocounter{yearind}{\value{cntind}}&
\bf\showcnt{cntsam}\addtocounter{yearsam}{\value{cntsam}}&
\bf\showcnt{cntpra}\addtocounter{yearpra}{\value{cntpra}}&
\bf\showcnt{cntxxx}\addtocounter{yearxxx}{\value{cntxxx}}&
\showcnt{cnttotal}\\
\hline

\end{longtable}
}
}



\hbox to \textwidth {\hbox{\bf \nextsection{} УЧЕБНАЯ РАБОТА}\hfil\hbox{\bf ПЛАН НА 1-й семестр}}
\tables 1 \uchebPLANa\uchebFAKTa

\newpage

\hbox to \textwidth {\hbox{\bf ПЛАН НА 2-й семестр}\hfil\hbox{}}
\tables 2 \uchebPLANb\uchebFAKTb

\begin{longtable}{|c|l|c|c|c|c|c|c|c|c|c|c|c|c|}
\phantomrow \kill
\hline
\multicolumn{3}{|l|}{\bf Итого за год:} &
\arabic{yearlek}&
\arabic{yearsem}&
\arabic{yearlab}&
\arabic{yearzac}&
\arabic{yearekz}&
\arabic{yearkon}&
\arabic{yearind}&
\arabic{yearsam}&
\arabic{yearpra}&
\arabic{yearxxx}&
\addtocounter{yeartotal}{\value{yearlek}}%
\addtocounter{yeartotal}{\value{yearsem}}%
\addtocounter{yeartotal}{\value{yearlab}}%
\addtocounter{yeartotal}{\value{yearzac}}%
\addtocounter{yeartotal}{\value{yearekz}}%
\addtocounter{yeartotal}{\value{yearkon}}%
\addtocounter{yeartotal}{\value{yearind}}%
\addtocounter{yeartotal}{\value{yearsam}}%
\addtocounter{yeartotal}{\value{yearpra}}%
\addtocounter{yeartotal}{\value{yearxxx}}%
\bf\arabic{yeartotal}
\\
\hline
\end{longtable}

\begin{flushright}
    ({\em План на год: \arabic{cntuchebtotal}})
\end{flushright}
\vfil

\vspace{1em}
\hbox to \textwidth {\hbox{\date}\hfil\hbox{Подпись преподавателя\und}}
\vspace{2em}
\hbox to \textwidth {\hbox{Выполнение утверждено на заседании кафедры}\hfil\hbox{\protokol}}


\addtocounter{cnttotaltotal}{\value{cntuchebtotal}} % глобальный счётчик ваще всё! (plan.tex)

