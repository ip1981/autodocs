\subsection{Научная проблема и конкретная задача в ее рамках, на решение которой направлен проект}
Википедия --- свободная общедоступная многоязычная универсальная энциклопедия,
поддерживаемая некоммерческой организацией <<Фонд Викимедиа>>. Миллионы
статей Википедии написаны совместно добровольцами со всего мира, и
все эти статьи могут быть изменены кем угодно, кому доступен сайт
Википедии. Википедия сейчас является самым крупным и наиболее популярным
справочником в Интернете. По объёму сведений и тематическому охвату
считается самой полной энциклопедией из когда-либо создававшихся за
всю историю человечества. Одним из основных достоинств Википедии как
универсальной энциклопедии является возможность представить информацию
на родном языке, сохраняя её ценность в аспекте культурной принадлежности.~---
\href{http://ru.wikipedia.org/wiki/\%D0\%92\%D0\%B8\%D0\%BA\%D0\%B8\%D0\%BF\%D0\%B5\%D0\%B4\%D0\%B8\%D1\%8F}{http://ru.wikipedia.org/wiki/Википедия}.

Русская Википедия (русский раздел Википедии) значительно уступает
английской по количеству статей и охвату материала, а количество и
качество статей о точных науках, таких как физика и математика, следует
признать неудовлетворительными.%
\footnote{\href{http://habrahabr.ru/blogs/wikipedia/73275/}{http://habrahabr.ru/blogs/wikipedia/73275/}%
} Отсутствуют статьи по некоторым свежим темам, например, по ауксетикам,%
\footnote{\href{http://en.wikipedia.org/wiki/Auxetics}{http://en.wikipedia.org/wiki/Auxetics}%
} техническим единицам, вроде ANSI Lm.%
\footnote{\href{http://en.wikipedia.org/wiki/Lumen_\%28unit\%29}{http://en.wikipedia.org/wiki/Lumen\_{}(unit)}%
} Изложение некоторых имеющихся статей вряд ли годится для универсальной
(читай: популярной) энциклопедии, а носит справочный характер, не
вскрывающий суть описываемого явления.

Целью проекта является написание новых статей в русской Википедии,
правка имеющихся русскоязычных, переводы английских статей по физике
и математике.


\subsection{Актуальность предлагаемых исследований
для данной отрасли знаний и социально-экономическое значение проекта для Архангельской области}

Википедия всё чаще используется студентами и преподавателями для быстрого
введения в суть явления, проблемы или для обращения за исторической
справкой. Часто их интерес бывает не удовлетворён из-за отсутствия
нужной статью или из-за её краткости, или просто из-за слишком специального
изложения. Грамотные, полные и доходчивые статьи не только пойдут
на пользу молодёжи, но и поднимут научный и образовательный престиж
Архангельской области в глазах и мониторах мирового сообщества.


\subsection{Новизна поставленной задачи}
Впервые за историю человечества информация может быть так легко доступна,
как в Википедии.


\subsection{Предлагаемые методы и подходы, общий план работы}
Работа будет заключаться в написании новых статей, переводе англоязычных
статей, дополнении и исправлении имеющихся статей по физическим и
математическим темам, имеющим прямое или косвенное отношение к учебным
программам ПГУ.

\subsection{Ожидаемые конкретные научные результаты (форма изложения должна дать возможность провести экспертизу результатов и оценить степень выполнения заявленного в проекте плана работы)}
Все действия исполнителя будут записаны на сайте Википедии. Вклад
участника Википедии отображается на специальной странице.%
\footnote{\href{http://ru.wikipedia.org/wiki/\%D0\%A1\%D0\%BB\%D1\%83\%D0\%B6\%D0\%B5\%D0\%B1\%D0\%BD\%D0\%B0\%D1\%8F:Contributions/Igor_Pashev}{http://ru.wikipedia.org/wiki/Служебная:Contributions/Igor\_{}Pashev}%
}

\subsection{Имеющийся у коллектива научный задел
по предлагаемому проекту (полученные ранее результаты, разработанные программы и методы)}

Исполнитель уже внёс вклад в Википедию, например, переведена статья
о физике Эмиле Вихерте.%
\footnote{\href{http://ru.wikipedia.org/wiki/\%D0\%92\%D0\%B8\%D1\%85\%D0\%B5\%D1\%80\%D1\%82,_\%D0\%AD\%D0\%BC\%D0\%B8\%D0\%BB\%D1\%8C}{http://ru.wikipedia.org/wiki/Вихерт,\_{}Эмиль}%
}

\subsection{Список основных публикаций коллектива, наиболее близко относящихся к предлагаемому проекту}
Публикации по проекту отсутствуют.

\subsection{Перечень оборудования и материалов, имеющихся у коллектива для выполнения проекта}
Служебный персональный компьютер с выходом в Интернет.

