\documentclass[12pt,russian]{article}
\usepackage{ifpdf}
\ifpdf
\usepackage{cmap}
\fi
\usepackage[T2A]{fontenc}
\usepackage[utf8]{inputenc}
\usepackage[a4paper]{geometry}
\geometry{verbose,tmargin=2cm,bmargin=2cm,lmargin=2cm,rmargin=2cm}
\usepackage{babel}

\usepackage{multirow}
\usepackage{calc}
\usepackage{icomma}
\usepackage{array}
\usepackage{indentfirst}
\usepackage{booktabs}
\usepackage{amsmath}
\usepackage{amssymb}
\usepackage{setspace}
\usepackage[unicode=true, pdfusetitle,
 bookmarks=true,bookmarksnumbered=false,bookmarksopen=false,
 breaklinks=true,pdfborder={0 0 0},backref=page,colorlinks=false]
 {hyperref}

\renewcommand{\thefootnote}{\arabic{footnote})}


\sloppy
\widowpenalty=10000
\clubpenalty=10000
\raggedbottom
\frenchspacing
\setlength\parskip{\medskipamount}
\setlength\parindent{1.5cm}

\lccode`\-=`\-
\defaulthyphenchar=127


% Пометка к полю формы
\usepackage{color}
\definecolor{gray}{rgb}{0.4,0.4,0.4}
\newcommand{\formhint}[1]{\textcolor{gray}{\sffamily\tiny #1}}
\newcommand{\formfield}[3]{\parbox{#1}{\parbox{#1}{\setlength{\baselineskip}{0.6em}{\formhint{#2:}}}\\#3\bigskip}}
\newcommand{\blankfield}[3]{\formfield{#1}{#2}{\phantom{#3}}}

\def\sign{\begin{flushright}Подпись руководителя проекта:
    \underline{~~~~~~~~~~~~~~~~~~~~~~~~~~~~~~~~~~~~~~~~~~}\end{flushright}}


% Ширина поля в титульной таблице
\newlength{\col}
\setlength{\col}{0.48\textwidth}



% \replicate{10}{text} - повторить text 10 раз
\newcommand*\recur[1]{\csname rn#1\recur}
\newcommand\rnm[1]{\endcsname{#1}#1}
\newcommand\rn[1]{}
\newcommand*\replicate[1]{%
\csname rn\expandafter\recur
\romannumeral\number\number#1 000\endcsname\endcsname
}

% знак рубля
\def\rub{руб.}

% счётчик числа исполнителей
\def\maxworkers{10}
\newcounter{worker}

\newcounter{budget}
\newcounter{taxd}
\newcounter{taxm}


% Вычисление объёма финансирования
% Оплата труда штатных сотрудников организации
\def\ssalarycalc{Каждой тваре по паре}
\def\ssalary{340}

% Оплата труда внештатных сотрудников
\def\nsalarycalc{Откаты}
\def\nsalary{100}

% Приобретение предметов снабжения и расходных материалов
\def\materialscalc{Пассатижи}
\def\materials{20}

% Услуги сторонних организаций
\def\servicecalc{Массаж}
\def\service{30}

% Командировочные расходы
\def\tripcalc{Автобус}
\def\trip{40}

% Прочие текущие расходы и услуги
\def\othercalc{Еда}
\def\other{50}


\setcounter{budget}{0}
\addtocounter{budget}{\ssalary}
\addtocounter{budget}{\nsalary}


\setcounter{taxd}{\arabic{budget} * 23}
\setcounter{taxm}{\arabic{taxd} - (\arabic{taxd} / 100) * 100}
\setcounter{taxd}{\arabic{taxd} / 100}

\edef\taxcalc{$(\ssalary + \nsalary)\times23\%$}
\edef\tax{\arabic{taxd}\ifnum\value{taxm}>0,\arabic{taxm}\fi}


\addtocounter{budget}{\materials}
\addtocounter{budget}{\service}
\addtocounter{budget}{\trip}
\addtocounter{budget}{\other}
\addtocounter{budget}{\arabic{taxd}}
\edef\budget{\arabic{budget}\ifnum\value{taxm}>0,\arabic{taxm}\fi}




\begin{document}


\section*{Заявка на конкурс <<Молодые учёные Поморья>>}
\begin{center}


% Название проекта
\def\title{Точные науки в Википедии}

% Вид конкурса: А или Б
\def\type{А}

% Область знаний: 1, 2, 3, 4
\def\domain{3, 4}


% *** Организация, где выполняется проект ***

% Полное название организации
\def\organization{Поморский государственный университет им.~М.~В.~Ломоносова}

% Краткое название организации
\def\org{ПГУ}

% Фамилия, имя, отчество руководителя организации
\def\director{Луговская, Ирина Робертовна}

% Телефон руководителя организации
\def\directorphone{68-27-80}


% *** Организация, через которую осуществляется финансирование ***

% Полное название организации
\def\forganization{Поморский государственный университет им.~М.~В.~Ломоносова}

% Краткое название организации
\def\forg{ПГУ}

% Фамилия, имя, отчество руководителя организации
\def\fdirector{Луговская, Ирина Робертовна}

% Телефон руководителя организации
\def\fdirectorphone{68-27-80}

% Фамилия, имя, отчество главбуха
\def\bookkeeper{}

% Телефон бухгалтерии
\def\bookkeeperphone{68-37-82}

% Пункт разрешения на открытие внебюджетного лицевого счета,
% в соответствии с которым  учреждение имеет право получать
% субсидии на выполнение научных проектов
\def\clause{\phantom{это писец!}}



 % основные данные проекта

% создание списка исполнителей

% Фамилия, имя, отчество (полностью)
\def\name{Пашев, Игорь Николаевич}

% Дата рождения
\def\birth{29 сентября 1981}

% Учёная степень
\def\degree{Кандидат физико-математических наук}

% Год присуждения учёной степени
\def\degreeyear{2006}

% Учёное звание
\def\rank{---}

% Год присвоения учёного звания
\def\rankyear{---}

% Полное название организации основного места работы
\def\worganization{Поморский государственный университет им.~М.~В.~Ломоносова}

% Сокращенное название организации основного места работы
\def\worg{ПГУ}

% Должность (с указанием кафедры, лаборатории и т.~д.)
\def\position{Страшный преподаватель кафедры теоретической физики}

% Область научных интересов (ключевые слова, не более 15)
\def\keywords{Физика, математика, информатика}

% Общее число публикаций
\def\npublications{Четырнадцать}

% Поддержка проектов заявителя в форме грантов
% (название других фондов, год, номер и название проекта)
\def\grants{---}

% Почтовый индекс
\def\index{163062}

% Почтовый адрес
\def\address{г.~Архангельск, ул.~Воронина, дом.~30, корп.~2.}

% Телефон служебный
\def\workphone{68-31-75}

% Телефон домашний
\def\homephone{---}

% Факс
\def\fax{---}

% Электронный адрес
\def\email{pashev.igor@gmail.com}

 % данные руководителя проекта
\edef\team{\name}
\setcounter{worker}{0}
\replicate{\maxworkers} % данные исполнителей проекта
{
    \stepcounter{worker}
    \IfFileExists{worker-\arabic{worker}.tex}
    {
        \input{worker-\arabic{worker}.tex}
        \edef\team{\team{}; \name{}}
    }
    {}
}


% Фамилия, имя, отчество (полностью)
\def\name{Пашев, Игорь Николаевич}

% Дата рождения
\def\birth{29 сентября 1981}

% Учёная степень
\def\degree{Кандидат физико-математических наук}

% Год присуждения учёной степени
\def\degreeyear{2006}

% Учёное звание
\def\rank{---}

% Год присвоения учёного звания
\def\rankyear{---}

% Полное название организации основного места работы
\def\worganization{Поморский государственный университет им.~М.~В.~Ломоносова}

% Сокращенное название организации основного места работы
\def\worg{ПГУ}

% Должность (с указанием кафедры, лаборатории и т.~д.)
\def\position{Страшный преподаватель кафедры теоретической физики}

% Область научных интересов (ключевые слова, не более 15)
\def\keywords{Физика, математика, информатика}

% Общее число публикаций
\def\npublications{Четырнадцать}

% Поддержка проектов заявителя в форме грантов
% (название других фондов, год, номер и название проекта)
\def\grants{---}

% Почтовый индекс
\def\index{163062}

% Почтовый адрес
\def\address{г.~Архангельск, ул.~Воронина, дом.~30, корп.~2.}

% Телефон служебный
\def\workphone{68-31-75}

% Телефон домашний
\def\homephone{---}

% Факс
\def\fax{---}

% Электронный адрес
\def\email{pashev.igor@gmail.com}

 % снова, чтобы переопределить переменные

\begin{tabular}{>{\raggedright}c|c}
    \toprule
    \multirow{3}{*}{\formfield{\col}{Название проекта}{\title}}
    &    \formfield{\col}{Номер проекта}{} \tabularnewline
    &    \formfield{\col}{Вид конкурса}{\type} \tabularnewline
    &    \formfield{\col}{Область знания}{\domain}\tabularnewline    
     \midrule 

     \multirow{2}{*}{\formfield{\col}{Фамилия, имя, отчество руководителя проекта}{\name}}
     & \formfield{\col}{Телефон руководителя проекта}{\workphone}\tabularnewline 
     & \formfield{\col}{Электропочта руководителя проекта}{\email}\tabularnewline
     \midrule

    \multirow{2}{*}{\formfield{\col}{Полное и краткое название организации, где выполняется проект}
    {\organization{} (\org)}}
    & \formfield{\col}{Фамилия, имя, отчество руководителя организации} {\director}\tabularnewline 
    & \formfield{\col}{Телефон руководителя организации}{\directorphone}\tabularnewline
    \midrule 

    \multirow{5}{*}{\formfield{\col}{Организация, через которую осуществляется финансирование}
        {\forganization{} (\forg)}}
    &    \formfield{\col}{Фамилия, имя, отчество  руководителя организации}{\fdirector}\tabularnewline
    &    \formfield{\col}{Телефон руководителя организации}{\fdirectorphone}\tabularnewline
    &    \formfield{\col}{Фамилия, имя, отчество главного бухгалтера}{\bookkeeper}\tabularnewline
    &    \formfield{\col}{Телефон бухгалтерии}{\bookkeeperphone}\tabularnewline
    &    \formfield{\col}{Пункт разрешения на открытие внебюджетного лицевого счета,
    в соответствии с которым  учреждение имеет право получать
    субсидии на выполнение научных проектов}{\clause}\tabularnewline
    \midrule

    \multicolumn{2}{l}{\formfield{\col}{Запрашиваемый объем финансирования (в руб.)}
    {\budget~\rub{}}}\tabularnewline    \midrule 

    \multicolumn{2}{l}{
    \formfield{\linewidth}{Фамилия, имя, отчество основных исполнителей}
    {\team{}}
    }
    \tabularnewline
    \midrule

    \formfield{\col}{Подпись руководителя проекта}{}
    & \formfield{\col}{Дата подачи заявки}{\today}\tabularnewline
    \bottomrule
\end{tabular}
\end{center}



\newpage
\section{Содержание проекта}

\subsection{Научная проблема и конкретная задача в ее рамках, на решение которой направлен проект}
Википедия --- свободная общедоступная многоязычная универсальная энциклопедия,
поддерживаемая некоммерческой организацией <<Фонд Викимедиа>>. Миллионы
статей Википедии написаны совместно добровольцами со всего мира, и
все эти статьи могут быть изменены кем угодно, кому доступен сайт
Википедии. Википедия сейчас является самым крупным и наиболее популярным
справочником в Интернете. По объёму сведений и тематическому охвату
считается самой полной энциклопедией из когда-либо создававшихся за
всю историю человечества. Одним из основных достоинств Википедии как
универсальной энциклопедии является возможность представить информацию
на родном языке, сохраняя её ценность в аспекте культурной принадлежности.~---
\href{http://ru.wikipedia.org/wiki/\%D0\%92\%D0\%B8\%D0\%BA\%D0\%B8\%D0\%BF\%D0\%B5\%D0\%B4\%D0\%B8\%D1\%8F}{http://ru.wikipedia.org/wiki/Википедия}.

Русская Википедия (русский раздел Википедии) значительно уступает
английской по количеству статей и охвату материала, а количество и
качество статей о точных науках, таких как физика и математика, следует
признать неудовлетворительными.%
\footnote{\href{http://habrahabr.ru/blogs/wikipedia/73275/}{http://habrahabr.ru/blogs/wikipedia/73275/}%
} Отсутствуют статьи по некоторым свежим темам, например, по ауксетикам,%
\footnote{\href{http://en.wikipedia.org/wiki/Auxetics}{http://en.wikipedia.org/wiki/Auxetics}%
} техническим единицам, вроде ANSI Lm.%
\footnote{\href{http://en.wikipedia.org/wiki/Lumen_\%28unit\%29}{http://en.wikipedia.org/wiki/Lumen\_{}(unit)}%
} Изложение некоторых имеющихся статей вряд ли годится для универсальной
(читай: популярной) энциклопедии, а носит справочный характер, не
вскрывающий суть описываемого явления.

Целью проекта является написание новых статей в русской Википедии,
правка имеющихся русскоязычных, переводы английских статей по физике
и математике.


\subsection{Актуальность предлагаемых исследований
для данной отрасли знаний и социально-экономическое значение проекта для Архангельской области}

Википедия всё чаще используется студентами и преподавателями для быстрого
введения в суть явления, проблемы или для обращения за исторической
справкой. Часто их интерес бывает не удовлетворён из-за отсутствия
нужной статью или из-за её краткости, или просто из-за слишком специального
изложения. Грамотные, полные и доходчивые статьи не только пойдут
на пользу молодёжи, но и поднимут научный и образовательный престиж
Архангельской области в глазах и мониторах мирового сообщества.


\subsection{Новизна поставленной задачи}
Впервые за историю человечества информация может быть так легко доступна,
как в Википедии.


\subsection{Предлагаемые методы и подходы, общий план работы}
Работа будет заключаться в написании новых статей, переводе англоязычных
статей, дополнении и исправлении имеющихся статей по физическим и
математическим темам, имеющим прямое или косвенное отношение к учебным
программам ПГУ.

\subsection{Ожидаемые конкретные научные результаты (форма изложения должна дать возможность провести экспертизу результатов и оценить степень выполнения заявленного в проекте плана работы)}
Все действия исполнителя будут записаны на сайте Википедии. Вклад
участника Википедии отображается на специальной странице.%
\footnote{\href{http://ru.wikipedia.org/wiki/\%D0\%A1\%D0\%BB\%D1\%83\%D0\%B6\%D0\%B5\%D0\%B1\%D0\%BD\%D0\%B0\%D1\%8F:Contributions/Igor_Pashev}{http://ru.wikipedia.org/wiki/Служебная:Contributions/Igor\_{}Pashev}%
}

\subsection{Имеющийся у коллектива научный задел
по предлагаемому проекту (полученные ранее результаты, разработанные программы и методы)}

Исполнитель уже внёс вклад в Википедию, например, переведена статья
о физике Эмиле Вихерте.%
\footnote{\href{http://ru.wikipedia.org/wiki/\%D0\%92\%D0\%B8\%D1\%85\%D0\%B5\%D1\%80\%D1\%82,_\%D0\%AD\%D0\%BC\%D0\%B8\%D0\%BB\%D1\%8C}{http://ru.wikipedia.org/wiki/Вихерт,\_{}Эмиль}%
}

\subsection{Список основных публикаций коллектива, наиболее близко относящихся к предлагаемому проекту}
Публикации по проекту отсутствуют.

\subsection{Перечень оборудования и материалов, имеющихся у коллектива для выполнения проекта}
Служебный персональный компьютер с выходом в Интернет.


\sign
\newpage



\newcounter{i}
\newcommand{\myitem}{{\bf\arabic{i}.} \stepcounter{i}}

\def\info#1#2
{ % 1 - Р или И, 2 - руководителя или исполнителя
    \setcounter{i}{1}
    \setlength\parskip{0pt}
    \setlength\parindent{0cm}
    \formfield{\linewidth}{Фамилия, имя, отчество (полностью)}{\myitem \name}\par
    \formfield{\linewidth}{Дата рождения}{\myitem \birth}\par
    \formfield{\linewidth}{Учёная степень }{\myitem \degree}\par
    \formfield{\linewidth}{Год присуждения учёной степени}{\myitem \degreeyear}\par
    \formfield{\linewidth}{Учёное звание}{\myitem \rank}\par
    \formfield{\linewidth}{Год присвоения учёного звания}{\myitem \rankyear}\par
    \formfield{\linewidth}{Полное название организации основного места работы}{\myitem \worganization}\par
    \formfield{\linewidth}{Сокращенное название организации основного места работы}{\myitem \worg}\par
    \formfield{\linewidth}{Должность (с указанием кафедры, лаборатории и т.~д.)}{\myitem \position}\par
    \formfield{\linewidth}{Область научных интересов (ключевые слова, не более 15)}{\myitem \keywords}\par
    \formfield{\linewidth}{Общее число публикаций}{\myitem \npublications}\par
    \formfield{\linewidth}{Поддержка проектов заявителя в форме грантов
    (название других фондов, год, номер и название проекта)}{\myitem \grants}\par
    \formfield{\linewidth}{Почтовый индекс}{\myitem \index}\par
    \formfield{\linewidth}{Почтовый адрес}{\myitem \address}\par
    \formfield{\linewidth}{Телефон служебный}{\myitem \workphone}\par
    \formfield{\linewidth}{Телефон домашний}{\myitem \homephone}\par
    \formfield{\linewidth}{Факс}{\myitem \fax}\par
    \formfield{\linewidth}{Электронный адрес}{\myitem \email}\par
    \formfield{\linewidth}{Участие в проекте (буква <<Р>> --- руководитель; буква <<И>> --- исполнитель)}{\myitem #1}\par

    \formfield{\linewidth}{Подписи #2:}
    {\underline{\phantom{АААААААААААААААААААААААААААААААААААА}}}
}

\section{Сведения о руководителе проекта}


% Фамилия, имя, отчество (полностью)
\def\name{Пашев, Игорь Николаевич}

% Дата рождения
\def\birth{29 сентября 1981}

% Учёная степень
\def\degree{Кандидат физико-математических наук}

% Год присуждения учёной степени
\def\degreeyear{2006}

% Учёное звание
\def\rank{---}

% Год присвоения учёного звания
\def\rankyear{---}

% Полное название организации основного места работы
\def\worganization{Поморский государственный университет им.~М.~В.~Ломоносова}

% Сокращенное название организации основного места работы
\def\worg{ПГУ}

% Должность (с указанием кафедры, лаборатории и т.~д.)
\def\position{Страшный преподаватель кафедры теоретической физики}

% Область научных интересов (ключевые слова, не более 15)
\def\keywords{Физика, математика, информатика}

% Общее число публикаций
\def\npublications{Четырнадцать}

% Поддержка проектов заявителя в форме грантов
% (название других фондов, год, номер и название проекта)
\def\grants{---}

% Почтовый индекс
\def\index{163062}

% Почтовый адрес
\def\address{г.~Архангельск, ул.~Воронина, дом.~30, корп.~2.}

% Телефон служебный
\def\workphone{68-31-75}

% Телефон домашний
\def\homephone{---}

% Факс
\def\fax{---}

% Электронный адрес
\def\email{pashev.igor@gmail.com}


\info{Р}{руководителя}
\newpage

\section{Сведения об исполнителях проекта}
\setcounter{worker}{0}
\replicate{\maxworkers}
{
    \stepcounter{worker}
    \IfFileExists{worker-\arabic{worker}.tex}
    {
        \input{worker-\arabic{worker}.tex}
        \info{И}{исполнителя}
        \newpage
    }
    {}
}



\section{Смета расходов на выполнение проекта}

\begin{center}
\begin{tabular}{|>{\raggedright}m{0.4\textwidth}|>{\centering}m{0.4\textwidth}|r|}
\hline 
\textbf{Виды расходов} & \textbf{Расчёт} & \textbf{Сумма}\tabularnewline
\hline
\hline 
Оплата труда штатных сотрудников организации & \ssalarycalc & \ssalary~\rub{}\tabularnewline
\hline 
Оплата труда внештатных сотрудников & \nsalarycalc & \nsalary~\rub{} \tabularnewline
\hline 
Начисления на заработную плату & \taxcalc & \tax~\rub{}\tabularnewline
\hline 
Приобретение предметов снабжения и расходных материалов & \materialscalc & \materials~\rub{} \tabularnewline
\hline 
Услуги сторонних организаций & \servicecalc & \service~\rub{} \tabularnewline
\hline 
Командировочные расходы & \tripcalc & \trip~\rub{} \tabularnewline
\hline 
Прочие текущие расходы и услуги & \othercalc & \other~\rub{} \tabularnewline
\hline
\hline 
\textbf{Общая сумма расходов} &  & \textbf{\budget~\rub{}}\tabularnewline
\hline
\end{tabular}
\end{center}


\sign

\end{document}

